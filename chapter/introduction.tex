\chapter*{前\hspace{1em}言}
\markboth{前言}{前言}

时光荏苒,本科四年的生活即将结束。当年那个风华正茂的少年初次来到求是园时,就毫不犹豫地选择了光电专业。我热爱光学,想要在未来投身科研事业。四年过去,一路坎坷。岁月带走了我心中曾有过的激情,留下的只有那份看淡一切的平静。光电专业本身属于偏科研的专业,本科学到的只是浅层的知识,如果想要掌握真正的专业技能,必须读研深造。然而当今的国际局势风云变幻,许多原本打算出国留学的学生选择保研,各个专业都开始内卷化,让我们这些水平普普通通的学生的读研道路难上加难。经过深思熟虑后,那天晚上,我下定决心离开求是园,回到熟悉的北方求学。

这本书诞生于我为考研奋战的过程中。浙江大学光电学院的本科专业核心课程主要有三门,分别是应用光学、物理光学和光电子学。在2019年末,我顺利完成了本科光电专业必修理论知识的学习。为了方便查阅光学方面的各种基本概念,复习巩固光学基础知识,我便开始计划写下本书。本书进行书写时参考的资料主要为国内光电专业通用的课程教材:郁道银等编著的《工程光学(第四版)》和李晓彤等编著的《几何光学·像差·光学设计(第三版)》。我将本书看作是一本工具书,我对这本工具书的要求是,文字内容源于教材但要凝练,读者能够方便查阅相关概念,配图要精美,排版要美观。文字内容方面,因为本书的面向对象是光电专业的读者,因此对于一些简单的光学概念本书不再进行原理分析,本书重点突出了一些我认为需要重点掌握和易错的知识点。为了方便查阅,本书对光学概念分章节进行汇总,并设有目录,文中的图表、公式都设有标签可以进行跳转。配图方面,本书尽可能多地使用矢量图,这些矢量图来自维基百科和我自己的绘制。但因为我的时间有限,仍有较多图片使用了教材中的截图。对于本书的排版,我采用了\LaTeX 进行写作,模板采用的是Elegant\LaTeX 系列模板中的 \href{https://github.com/ElegantLaTeX/ElegantBook}{ElegantBook}。本书为免费共享书籍,请勿用于商业用途。

\begin{flushright}
	{\kaishu{关其锐\qquad\qquad\ }}
	\par 2021年6月15日于求是园
\end{flushright}